% Chapter Template

\chapter{Introducción específica} % Main chapter title

\label{Chapter2} % Change X to a consecutive number; for referencing this chapter elsewhere, use \ref{ChapterX}

%----------------------------------------------------------------------------------------
%	SECTION 1
%----------------------------------------------------------------------------------------
En el presente capítulo se introducen los módulos principales del equipo dip coater fabricado.   

\section{Movimientos controlados}
\subsection{Estudio preliminar}

Para entender la relación entre la velocidad de extracción y el espesor de material depositado se tuvo en consideración la siguiente publicación \cite{paper_galo}, que describe la técnica dip coating como un proceso dinámico complejo difícil de modelar, debido a los gradientes de concentración y viscosidad generados por evaporación de la solución. 


La publicación se basa entonces en un estudio semi-experimental sobre varias soluciones químicas para predecir el espesor final de la película. Tiene en cuenta dos modelos matemáticos, un modelo de capilaridad asociado a extracciones en velocidades bajas y otro modelo de evaporación asociado a velocidades altas. 

Se observa en la figura \ref{fig:paper_galo} la variación de los espesores fabricados respecto a la velocidad utilizada, también se puede observar como se relacionan los diferentes modelos. 

\begin{figure}[ht]
\centering 
\includegraphics[width=0.6\textwidth]{./Figures/paper_galo.png}
\caption{Espesor vs velocidad. Imagen tomada de \cite{paper_galo}.}
\label{fig:paper_galo}
\end{figure}

Los resultados del experimento concluyen  que existe linealidad  en la relación de espesor respecto la velocidad de extracción entre \SI{60}{\milli\meter\per\minute} y \SI{600}{\milli\meter\per\minute} . En donde el fenómeno se puede explicar por el modelo de evaporación.

Se desprende de este análisis la importancia de los siguientes requerimientos funcionales definidos por el cliente: 

\begin{itemize}
\item El sistema debe contar con un rango de velocidades de desplazamiento de muestra entre [1- 1000 \si{\milli\meter\per\minute}]. 
\item El sistema debe contar con un rango de aceleraciones de desplazamiento de muestra entre [1000 - 15000 \si{\meter\per\square\minute}].
\item El sistema debe permitir que el usuario pueda configurar en un programa variables de desplazamiento y tiempos de espera.		
\end{itemize}
	
Cabe destacar que todos los experimentos en la publicación fueron a realizados velocidad constante. De las reuniones con el cliente y del interés de trabajar en la frontera de la ciencia surgió la necesitad de poder darle al usuario la posibilidad de realizar experimentos a velocidad y aceleración controlada. Esto último es una cualidad que diferencia a nuestro equipo de todos los equipos comerciales relevados.

 
%Sin embargo, es una técnica muy difundida porque es simple y proporciona una excelente reproducibilidad. 
%El problema con este modelo es que la mayoría de las soluciones utilizadas son fluidos no-newtonianos, %es decir en donde el solvente de la solución se va evaparonado en simultáneo con la extracción de la %muestra induciendo una modificación en la densidad, tensión superficial y viscosidad del fluido. 
%Existen modelos matemáticos basados en la mecánica newtoniana que no tienen en cuenta la evaporación de %las soluciones y requieren varias suposiciones y simplificaciones. En estos modelos llegar a la %predicción del espesor depende de la densidad, la tensión superficial y la viscosidad del fluido. 
%La importancia de estos resultados es que el rango de velocidades quedá incluido dentro de los %requerimientos de nuestro equipo. 


\subsection{Circuitos Integrados Trinamic}

Del de los siguientes requerimientos funcionales acordados con el cliente:
			
\begin{itemize}
\item El equipo deberá contar para realizar los movimientos con un motor paso a paso Nema 17.
\item Se utilizará un driver de motor de la marca TRINAMIC.
\end{itemize}

Surge la necesidad de trabajar con el fabricante de circuitos integrados \textit{TRINAMIC Motion Control}\citep{3_web_trinamic}. Como su nombre lo indica Trinamic se especializa en la fabricación de CI (circuitos integrados) para el control de diferentes tipos de motores, su lema se basa en convertir señales digitales en movimientos controlados. Tiene una experiencia de veinte años en la industria del control de motores, actualmente fue adquirida por la compañía Analog Devices.

Cabe destacar que los integrados fabricados por la empresa Trinamic se utilizan en diversas aplicaciones en donde la precisión es importante, como por ejemplo impresión 3D, automatización industrial, robótica y equipos de laboratorio médico entre otras.
Cuenta con una amplia gama de productos que se diferencian principalmente según el tipo de motor que se quiera accionar. En nuestro caso como ya tenemos defino la utilización de un motor paso a paso vamos a trabajar con el ultimó integrado diseñado para tal fin, el mismo es el TMC5130. 
  
Todos los CI requieren una configuración inicial de parámetros, que depende del tipo de  motor y de la carga asociada al mismo. Es por eso que la empresa ofrece el software TMCL-IDE  y diferentes placas de desarrollo para ayudar a realizar una correcta configuración de parámetros. La placa de desarrollo para este integrado es la \textit{TMC5130-Eval Evaluation Board}.

El TMCL-IDE se ejecuta sobre una placa de desarrollo general compatible con diferentes kits de evaluación. En la figura \ref{fig:tmc5130_placa} podemos observar a izquierda la placa Startrampe  que se conecta entre la computadora y la placa de evaluación del CI TMC5130 que se observa a derecha.

\begin{figure}[htpb]
\centering 
\includegraphics[width=0.9\textwidth]{./Figures/tmc5130_placa.png}
\caption{Placa de desarrollo Startrampe + placa de evaluación TMC5130.}
\label{fig:tmc5130_placa}
\end{figure}


  
\subsection{Driver TMC5130}

El driver TMC5130 permite operar motores bipolares de dos fases comúnmente conocidos como motores paso a paso. El CI incorpora una etapa de potencia con tecnología \textit{MOSFET (metal oxide semiconductor field effect transistor)}  que permite manejar corrientes de hasta dos amperios por fase. En el caso de nuestro dip coater el peso de la carga es despreciable, por lo tanto la corriente es suficiente. Se realizarán en el \ref{Chapter4} los respectivos ensayos.

Podemos observar en la figura \ref{fig:tmc5130_diagrama} el diagrama en bloque del CI.

\begin{figure}[htpb]
\centering 
\includegraphics[width=1.1\textwidth]{./Figures/tmc5130_diagrama.png}
\caption{Diagrama en bloques TMC5130.}
\label{fig:tmc5130_diagrama}
\end{figure}

La comunicación con el CI se puede establecer a través del protocolo \textit{ UART (Universal Asynchronous Receiver-Transmitter)} o \textit{SPI (Serial Peripheral Interface)},en el caso de nuestro equipo utilizaremos la comunicación SPI.


Una característica importante a destacar es la posibilidad de programar la cantidad de pasos que da el motor. Los pasos están relacionados con las fases  y con la cantidad de dientes que tiene el rotor y estator del motor. Un paso es el movimiento mínimo que el motor puede hacer. Un motor paso a paso, como su nombre lo indica, realiza movimientos a través de pasos sucesivos. Por ejemplo es común contar con algún motor en donde la especificación dice que el paso es de (\ang{1.8}), esto significa que por cada vuelta de motor (\ang{360}) el motor realizará 200 pasos.

Una funcionalidad que incorpora este CI es incrementar la cantidad de pasos, el fabricante los denomina micropasos, el driver puede generar hasta un máximo de 256 micropasos por cada paso del motor. Siguiendo con el ejemplo recién presentado, para un motor de paso (\ang{1.8}) tendríamos en total 51200 micropasos como se observa en la ecuación \ref{eq:micro_pasos}.

\begin{equation}
	\label{eq:micro_pasos}
		(360/1.8) * 256 = 51200 \textup{ micropasos por revolución}
\end{equation}


Otra funcionalidad que se utilizará es \textit{Stallguard2}, una función de alta precisión que mide la fuerza contraelectromotriz generada en la bobinas del motor por cambios de carga en el eje. Como se observa en la figura \ref{fig:tmc5130_stallGuard2} el valor de stallguard se decrementa linealmente a medida que la carga en el eje del motor aumenta. En nuestro caso se utilizará esta funcionalidad para realizar un posicionamiento inicial que servirá de referencia para todos los desplazamientos posteriores. Cada vez que el equipo se enciende que ejecuta esta funcionalidad para buscar el respectivo cero de máquina.
     
\begin{figure}[htpb]
\centering 
\includegraphics[width=0.9\textwidth]{./Figures/tmc5130_stallguard2.png}
\caption{Función stallGuard2.}
\label{fig:tmc5130_stallGuard2}
\end{figure}

También se utilizará \textit{coolStep}, una función que a través de mediciones de carga en el eje del motor adapta automáticamente la corriente suministrada hacia las bobinas, aumentando la eficiencia como puede observarse en la figura \ref{fig:tmc5130_coolStep}, cuyo efecto reduce la energía según hojas de datos \citep{3_web_trinamic_producto} hasta un \SI{75}{\percent}. Incluso en aplicaciones en donde la carga es constante como es el caso de nuestro equipo.

\begin{figure}[htpb]
\centering 
\includegraphics[width=0.9\textwidth]{./Figures/tmc5130_coolstep.png}
\caption{Función coolStep.}
\label{fig:tmc5130_coolStep}
\end{figure}

Por último se analizará la función \textit{dcStep}, es un modo de conmutación automática que ajusta la velocidad del motor en caso de existir cierta sobrecarga del eje, es decir que si no puede mover la carga acoplada al eje con la velocidad establecida se ajusta a una velocidad menor para poder seguir en movimiento y no detenerse por completo. 


En el capítulo \ref{Chapter3} se darán detalles de las configuraciones finales del equipo.

\section{Interfaz de usuario}

Respecto a la interfaz usuario-máquina surgió en reuniones con el cliente la necesidad de contar con una interfaz moderna, que permita a un usuario dentro de un laboratorio configurar el equipo a pie de máquina. 

Dando así lugar al siguiente requerimiento:
\begin{itemize}
\item La configuración de la máquina debe poder realizarse a través de una pantalla táctil.	
\end{itemize} 

Se decidió trabajar con pantallas del tipo \textit{HMI (human machine interface)}, este tipo de pantalla incorpora una unidad de procesamiento que se encarga exclusivamente del procesamiento gráfico. Cuentan en general con un software que permite crean pantallas utilizadas para control y configuración de equipos. 

En el caso de nuestro equipo el sistema embebido de control se comunicará con la pantalla HMI a través de del protocolo UART. 

Luego de una investigación de mercado se eligió a la empresa STONE \citep{web_stone}, el fabricante ofrece un catálogo amplio de equipos, en nuestro caso por las dimensiones finales del equipo se optó por una pantalla de 4.3 pulgadas. Se detalla en la siguiente tabla \ref{tab:tabla_stone} las características técnicas de dos pantallas del mismo fabricante:


\begin{table}[ht]
	\centering
	\caption[Comparación Stone]{Comparación pantallas táctiles Stone 4.3.}
	\begin{tabular}{l c c }    
		\toprule
		\textbf{}     & \textbf{STWI043WT} & \textbf{STVI043WT} \\
		\midrule
		CPU 			& 	Cortex A8         		& 	CortexM4 			 	\\		
		Refresh Rate    & 	1G Hz         			& 	200 MHz 				\\
		Image format  	& 	png,bmp,jpg,svg,gif     & 	bmp,jpg 				\\
		Resolution		& 	480×272 pixel	        & 	480×272 pixel 			\\
		Flash  			& 	256MB         			& 	128MB 					\\
		Color  			& 	262 K	          		& 	65 K 					\\
		PCB 			& 	2.0mm black, ROHS       & 	1.6mm green 			\\
		Touch Type		& 	Resistive    			& 	Resistive				\\
		Interface 		& 	RS232/RS422/RS485/TTL   & 	RS232/RS485/TTL			\\
		\bottomrule
		\hline
	\end{tabular}
	\label{tab:tabla_stone}
\end{table}

El modelo elegido fue el STWI043WT, que pertenece a la nueva línea productos, tiene mayor capacidad de procesamiento, cuenta con un software de configuración moderno con mayores funcionalidades que el anterior y el costo no supera el \SI{15}{\percent} respecto de su modelo predecesor.  


%El modelo elegido como se observa en la figura \ref{fig:stone}
%\begin{figure}[htpb]
%\centering 
%\includegraphics[width=0.5\textwidth]{./Figures/stone.png}
%\caption{Display táctil Stone.}
%\label{fig:stone}
%\end{figure}


\section{Equipo propuesto}
\subsection{Componentes electrónicos}


Se presentan a continuación los siguientes requerimientos de firmware:

\begin{itemize}

\item El sistema debe permitir que el usuario pueda configurar en un programa variables de desplazamiento y tiempos de espera.
\item Un programa previamente configurado debe poder ejecutarse o guardarse en memoria interna.
\item El usuario debe poder guardar al menos 10 programas en la memoria no volátil del sistema.
\item Se deberá utilizar un control de versionado de cambios durante el desarrollo del firmware.
\item El desarrollo del firmware debe realizarse con capas de abstracción de software de tal manera que permita en un futuro cambiar de microcontrolador sin mayor esfuerzo.
\item Deberá registrar variables de presión y temperatura [opcional].

\end{itemize}



Como se observa en la figura \ref{fig:equipo_propuesto}, el equipo dip coater  propuesto incluye los siguientes módulos: 

\begin{figure}[ht]
\centering 
\includegraphics[width=0.9\textwidth]{./Figures/cap2_esquema_propuesto.jpg}
\caption{Esquema propuesto.}
\label{fig:equipo_propuesto}
\end{figure}



\begin{enumerate}
\item Microcontrolador ESP32-WROOM. 
\item Periféricos principales (UART - SPI - I2C - FLASH Interna). 
\item Driver de motor paso a paso TMC5130.
\item Pantalla táctil STONE STWI043WT.
\end{enumerate}

\subsection{Componentes mecánicos}

Según los siguientes requerimientos asociados a las partes mecánicas: 

\begin{itemize}
\item La estructura principal del equipo debe ser fabricada con perfil de aluminio anodizado natural.
\item El recorrido mecánico de desplazamiento de muestra debe ser como mínimo de [3500 mm].
\item Las piezas especiales del equipo deben ser mecanizadas en aluminio.

\end{itemize}

Se decidió trabajar con el proveedor Perfiles de Aluminio .NET \citep{web_perfiles_net}, que cuenta con perfiles de diferentes modelos y dimensiones para construir la estructura.
El equipo también contará con una guía lineal acoplada al perfil principal. Para la elección de la guía se tuvieron en cuenta las siguientes consideraciones:

\begin{enumerate}
\item El ambiente cambia  según las soluciones químicas utilizadas. Es posible entonces que se trabaje con soluciones muy corrosivas que afecten la estructura.  
\item El uso de lubricantes en las guías podrían afectar la calidad del experimento.
\item Se deben evitar vibraciones en la estructura para no dañar la calidad del \textit{film}.

\end{enumerate}

Se decidió entonces trabajar con la empresa IGUS \citep{web_igus}, que se especializa en la fabricación de polímeros y ofrece guías lineales que se deslizan en lugar de rodar. Los polímeros están combinados con materiales anticorrosivos y no requieren la aplicación de lubricante, es decir que conforman un entorno de trabajo limpio y libre de mantenimiento periódico. Se observa en la figura \ref{fig:equipo_mecánico} cuatro tipos de guías en donde se puede apreciar el polímero auto-lubricado que se ubica entre el eje y el carro.

\begin{figure}[ht]
\centering 
\includegraphics[width=0.7\textwidth]{./Figures/guias.png}
\caption{Guía Lineal IGUS.}
\label{fig:equipo_mecánico}
\end{figure}

Para el diseño y fabricación de piezas mecanizadas en aluminio se trabajó con BOBCAD \citep{web_bobcad}, un software \textit{CAD/CAM (Computer-Aided Design /Computer-Aided Manufacturing )} utilizado en la industria manufacturera. 

Con la parte CAD diseñamos un primer modelo 3D de pieza y en nuestro caso antes de comenzar con el módulo CAM, realizamos  una impresión 3D con filamento PLA para probar las dimensiones y factibilidad de la pieza.
Una vez que el modelo queda aprobado comenzamos con el módulo CAM, este modulo se encarga de convertir el modelo 3D en lenguaje de máquina que el equipo puede interpretar. En el taller mecánico contamos con una Fresadora \textit{ CNC (computer numerical control)} de la marca FAGOR \citep{web_fagor} que interpreta G-CODE también conocido como RS-274 \citep{web_gcode}. 

En el capítulo \ref{Chapter3} se darán detalles mas precisos de piezas fabricadas.
 



