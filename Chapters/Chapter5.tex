% Chapter Template

\chapter{Conclusiones} % Main chapter title

\label{Chapter5} % Change X to a consecutive number; for referencing this chapter elsewhere, use \ref{ChapterX}


%----------------------------------------------------------------------------------------

%----------------------------------------------------------------------------------------
%	SECTION 1
%----------------------------------------------------------------------------------------
En esté capítulo se presentan las conclusiones principales que dejo la fabricación del un primer equipo dip coater, se resaltan los logros mas importantes del trabajo y de detallan algunos puntos que a mejorar en futuros trabajos, por último se plantean las planes inmediatos de comercialización del equipo.

\section{Resultados obtenidos }


EL principal hito del trabajo fue fabricar un MVP (producto mínimo viable) de equipo dip coater, que cuenta con las características suficientes para satisfacer a los primeros usuarios y será el primero diseñado y fabricado en argentina. 

Los logros fundamentales del trabajo fueron los siguientes:

\begin{itemize}
\item Placa electrónica.
\item Firmware modular que permite continuar el desarrollo ordenadamente.
\item Diseño mecánico propio y capacidad técnica desarrollada para fabricarlo. 
\end{itemize} 
 
 
Lamentablemente la planificación original no pudo ser sostenida, abarcar íntegramente la fabricación de un equipo fue demasiado trabajo para el tiempo establecido, desarrollar las partes mecánicas llevo un trabajo extra que no fue contemplado. Si bien fue un camino  



 



%----------------------------------------------------------------------------------------
%	SECTION 2
%----------------------------------------------------------------------------------------
\section{Próximos pasos}

Se plantean puntos fundamentales para el futuro inmediato del equipo.  

\begin{itemize}

\item Ensayos del MPV con potenciales usuarios, se pretende distribuir cinco equipos en diferentes laboratorios 
\item Ensayos del equipo con profesionales del áreas de las nanociencias.
\item Respecto al hardware ya se cuenta con una version estable en funcionamiento que sin embargo tendrá cambios cuando se mande a fabricar un nuevo lote. 
\item Firmware:
\item Diseño mecánico:
\item Comercialización:
\end{itemize}


